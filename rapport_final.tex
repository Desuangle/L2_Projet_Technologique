\documentclass[12pt]{article}
%\usepackage{graphicx}
\usepackage[utf8]{inputenc}
\usepackage[frenchb]{babel}
%\usepackage[]{geometry}
\usepackage{listings}
\lstset{breaklines=true}

\begin{document}

\title{Projets technologiques - Rapport final - Groupe TMA1C}
\author{Walid Alouini, Mathis Gendron, Killian Le Guen, Alexandre Vialar}
\date{\today}

\maketitle
\tableofcontents

\section{Git}
\subsection{Synthèse}
Pour notre gestion de projet nous avons utilisé Git et le serveur Savane du Cremi afin de pouvoir partager et accéder a notre projet de n'importe ou.
Git nous à aussi permis d'avoir un versionning de notre projet afin de revenir en arrière en cas de besoin.
Nous nous sommes assurés de ne push que des versions qui compilent.
\subsection{Analyse}
Lors de notre projet il nous est arrivé d'avoir des conflits de versions, afin de résoudre ce problème nous avons créé des branches locales. 
Ce qui à permit de pouvoir utilé git pull sur la branche master et de fusionner plus façilement avec la branche local.
\subsection{Bilan}
Lors du projet nous avond été amené a utilisé les fonctions : Git clone,Git init, pull, push, merge,branch,checkout
\subsection{Amélioration}
En plus de créer des branches locales nous aurions pu créer des branches distantes afin de partager nos versions incomplétes sans inpacter la branche master.

\section{Test}
\subsection{Synthèse}
Afin de vérifier le bon fonctionnement des fonctions de game.c nous avons crée un ensemble de test qui une fois lancé renvoie failure ou success 
pour chaque fonction testé
\subsection{Analyse}
Nous nous somme ainsi répartis les fonctions en 4 et nous avons ensuite chacun crée notre fichier.
Il fut parfois difficile de trouver tous les bugs possibles de nos fonctions car ne disposant que du game.o
nous n'avions pas accès au code sources et n'avions pas forcement une compréhension complète des fonctions, 
de plus certaines fonction sont dépendantes des autres ainsi elles envoie des faux positifs
Afin d'avoir des test cohérents nous avons suivit le standard fournit le fichier fournit.
\subsection{Bilan}
Nous avons testé toute les fonctionnalitées de la V1. Ainsi que les fonctions  is_wrapping, new_game_ext,new_game_empty_ext.
Certaines de nos fonctions de test ont aussi été mise à jour pour la v2.
\subsection{Amélioration}
Afin d'améliorer nos test il faudrait penser à ajouter les nouvelles fonctions développer et à minimiser les dépendances des tests

%\lstinputlisting[language=c]{solve_main.c}

\end{document}
