\documentclass[12pt]{article}
%\usepackage{graphicx}
\usepackage[utf8]{inputenc}
\usepackage[frenchb]{babel}
%\usepackage[]{geometry}
\usepackage[T1]{fontenc}
\usepackage{listings}
\lstset{breaklines=true}

\begin{document}

\title{Projets technologiques - Rapport final - Groupe TMA1C}
\author{Walid Alouini, Mathis Gendron, Killian Le Guen, Alexandre Vialar}
\date{\today}

\maketitle
\tableofcontents



\section{Makefile}
\subsection{Synthèse}

Réalisation d'un fichier Makefile pour la V1 afin d'automatiser la compilation des éxecutables du jeu et des différents tests.

\subsection{Analyse}

Nous avons utilisé les connaissances acquises lors du TD sur make dés la première version
de notre Makefile pour la V1, celle-ci faisant déjà usage de variables implicites (FLAGS) et automatiques.
Nous avons subséquemment ajouté le code nécessaire à la compilation des tests sur la V1 puis du nouveau
framework de test.

Nous n'avons rencontré aucun problème majeur dans l'implémentation de make, les seuls
problèmes mineurs pouvait provenir d'erreurs humaines dans l'écriture du Makefile et
nous n'avions aussi pas immédiatement compris qu'il fallait utiliser la commande \textbf{ar}
pour générer les librairies.

\subsection{Bilan}

L'utilisation de make a effectivement rendu la compilation et les tests associés au projet plus faciles 
(le nouveau framework de test permettant d'effectuer tous les tests avec une simple utilisation de la 
commande \textbf{make test}), aucun bug notoire n'a été associé à make lors du développement.

\subsection{Améliorations}

Les améliorations concernant make ont été apportée par l'utilisation de cmake qui permettait la 
génération automatique de Makefile.



\section{Net\_text}
\subsection{Synthèse}

Nous avons avec net\_text implémenté une version textuelle du jeu net via le terminal en demandant
seulement les coordonnées de la pièce à tourner à l'utilisateur.

\subsection{Analyse}

La première implémentation de net\_text employait un switch case pour associer chaque couple 
pièce, orientation à un caractère donné, de plus la gestion de l'interaction utilisateur 
était minimale et l'envoie de lettres en guise de coordonnées causait une boucle infinie.
Le code a été allégé en utilisant des tableaux de caractères à la place de switch case et 
la boucle infinie est désormais évitée en s'assurant que le programme va jusqu'à la fin de 
la chaîne de caractères fournie par l'utilisateur lorsque des coordonnées lui sont demandées.
Il reste cependant des légers problèmes : 
\begin{itemize}
\item L'entrée d'une seule coordonnée n'est pas considérée comme invalide et l'utilisateur peut indiquer la seconde coordonnée dans la ligne suivante du terminal.
\item Dut à la résolution de la boucle infinie, des coordonnées sous forme <nombre> <lettre> réaffichent la grille mais pas des coordonnées de la forme <nombre> <lettre>.
\end{itemize}

\subsection{Bilan}

En l'état le jeu en version texte est fonctionnel, la grille de jeu est affichée puis des 
les coordonnées d'une pièce à bouger sont demandées à chaque mouvement et le bug le plus 
important a été résolu. Aucun problème de performance n'est à déplorer.

\subsection{Améliorations}

Une meilleure gestion de l'interaction utilisateur est une bonne piste d'amélioration.
Il serait par exemple possible de préciser de ne pas utiliser de caractères autres que
des chiffres dans les coordonnées le cas échéant, il serait également possible de rendre
la lecture de la grille plus facile en indiquant les coordonnées directement sur celle-ci 
plutôt que de laisser l'utilisateur les trouver.



\section{Git}
\subsection{Synthèse}
Pour notre gestion de projet nous avons utilisé Git et le serveur Savane du Cremi afin de pouvoir partager et accéder a notre projet de n'importe ou.
Git nous à aussi permis d'avoir un versionning de notre projet afin de revenir en arrière en cas de besoin.
Nous nous sommes assurés de ne push que des versions qui compilent.
\subsection{Analyse}
Lors de notre projet il nous est arrivé d'avoir des conflits de versions, afin de résoudre ce problème nous avons créé des branches locales. 
Ce qui à permit de pouvoir utilé git pull sur la branche master et de fusionner plus façilement avec la branche local.
\subsection{Bilan}
Lors du projet nous avond été amené a utilisé les fonctions : Git clone,Git init, pull, push, merge,branch,checkout
\subsection{Amélioration}
En plus de créer des branches locales nous aurions pu créer des branches distantes afin de partager nos versions incomplétes sans inpacter la branche master.



\section{Test}
\subsection{Synthèse}
Afin de vérifier le bon fonctionnement des fonctions de game.c nous avons crée un ensemble de test qui une fois lancé renvoie failure ou success 
pour chaque fonction testé
\subsection{Analyse}
Nous nous somme ainsi répartis les fonctions en 4 et nous avons ensuite chacun crée notre fichier.
Il fut parfois difficile de trouver tous les bugs possibles de nos fonctions car ne disposant que du game.o
nous n'avions pas accès au code sources et n'avions pas forcement une compréhension complète des fonctions, 
de plus certaines fonction sont dépendantes des autres ainsi elles envoie des faux positifs
Afin d'avoir des test cohérents nous avons suivit le standard fournit le fichier fournit.
\subsection{Bilan}
Nous avons testé toute les fonctionnalitées de la V1. Ainsi que les fonctions is\_wrapping,new\_game\_ext,new\_game\_empty\_ext.
Certaines de nos fonctions de test ont aussi été mise à jour pour la v2.
\subsection{Amélioration}
Afin d'améliorer nos test il faudrait penser à ajouter les nouvelles fonctions développer et à minimiser les dépendances des tests



\section{game.c}
\subsection{Synthèse}
Ce fichier contient les définitions des fonctions décrivant l'interface de programmation du jeu.
\subsection{Analyse}
Définir tout ce qu'on a besoin dans la structure game\_s.

Parmit les difficultés rencontrées c'était dans la fonction bool is\_game\_over, on a eu quelques fuites mémoires.

\subsection{Bilan sur les fonctionnalités}
A part les fonctions principals dans le fichier game.h on a ajouté ces fonctions auxiliaires:

bool is\_connected\_coordinates

bool all\_pieces\_connected

void aux\_all\_pieces\_connected



\section{Cmake}
\subsection{Synthèse}
On a créer ce fichier Cmake afin de remplacer le fichier Makefile.
\subsection{Analyse}
Notre fichier CMakeLists.txt, va produire le script de compilation permettant la création de l'exécutable,
il est indépendant de la plateforme, et décrit comment compiler le projet à l'aide d'informations comme : le langage utilisé, les fichiers à compiler, les dépendances.

Ainsi CMakeLists.txt va pouvoir produire la bibliotheque game crée à partir de game.c et game\_io.c contient les fonctionalités du jeu

Parmit les difficultés rencontrées c'était quelques problémes syntaxique du Cmake.
\subsection{Bilan sur les fonctionnalités}
création des executables.
 
création de la bibliotheque game.
 
création des executables des tests du jeu.



\section{game\_io.c}
\subsection{Synthèse}
Ce fichier fournit des fonctions pour charger ou enregistrer un jeu(deux fonctions principale Load\_game et Save\_game), crée le jeu en chargeant sa description dans un fichier et retourner le jeu chargé.
\subsection{Analyse}
Notre but était de load et save un game a partir d'un fichier passer en paramétre, notre stratégie d'attaque du problème consiste a la création de quelques fonctions auxiliaires: convert\_piece a le rôle de convertir les caracteres pieces du fichier passer en paramétres en pieces (cette fonction appeler par la fonction load\_game), convert\_direction a le rôle de convertir les caracteres direntions en directions (cette fonction appeler par la fonction load\_game) et la fonction print\_piece a le rôle de convertir les pieces en caracteres(cette fonction appeler par la fonction save\_game).

Parmit les difficultés rencontrées c'était lors de load game, dans la première ligne du fichier passer en paramétre, le soucis est comment différencier les dérniers caractéres du mode wrapping et les caractéres piéces du jeu
\subsection{Bilan sur les fonctionnalités}
Fonction load\_game

Fonction save\_game

Fonction auxiliaire convert\_piece

Fonction auxiliaire convert\_direction

Fonction auxiliaire print\_piece

Fonction auxiliaire error
 


\section{Solveur}
\subsection{Synthèse}
On a fait une bibliothèque ``solve'' comprenant ``solve.c'' et ``solve.h'', ainsi qu'un fichier ``solve\_main.c'' qui permet de compiler le programe ``net\_solve''.

Le solveur renvoie soit une solution, soit le nombre de solutions, soit toutes les solutions pour un jeu donné.

On l'a intégré à la version SDL du jeu.
\subsection{Analyse critique}
On a fait un brainstorming après la présentation du solveur, et on en a sorti un algorithme. Après implémentation il ne fonctionnait pas, et on a analysé l'algorithme pour voir ce qui n'allait pas puis on l'a corrigé (L'algorithme final est trouvable en annexe~\ref{algosolv}).

Utiliser un algorithme était un bon choix car il permettait d'avoir une vue d'ensemble et il était facile de voir les erreurs à corriger.
\subsection{Bilan}
Le solveur fonctionne pour des jeux dont la taille va jusqu'à 20x20 (pour du 20x20 il met 5 secondes).

Il n'est pas très optimisé mais il est clair et c'est facile d'ajouter des options comme fait en tp noté.
\subsection{Pistes d'amélioration}
On pourrait revoir l'algorithme pour atteindre la fin de la récursion si et seulement si le jeu est finit. Ainsi il n'y aurait plus besoin de vérifier avec la fonction isgameover, ce qui améliorerait grandement la vitesse.



\section{Net SDL}
\subsection{Synthèse}
On a créé deux fichiers ``net\_sdl.c'' et ``main.c'', qui permettent de faire un rendu graphique du jeu grâce à la bibliothèque SDL2.
\subsection{Analyse critique}
Pour le ``main.c'' on a gardé celui de la démo, car il n'y avait rien à changer.

Pour ``net\_sdl.c'', on a repris ``model.c'' et on l'a rempli en s'inspirant de ``demo.c''.

On a rencontré des difficultés pour définir les variables globales nécessaires, au début on pensait utiliser des matrices de coordonnées ce qui est inutile.

Enfin on a testé assez tard le fait de viser en dehors du cadre, surtout du fait que pendant un certain temps c'était impossible car le jeu prenait la totalité du cadre, donc on n'a que tardivement corrigé un bug qui faisait planter le jeu en visant  à côté.
\subsection{Bilan}
Le jeu fonctionne, on a intégré le chargement de fichier comme dans net\_text, et on a aussi intégré le solveur.
\subsection{Pistes d'amélioration}
Pris par la pression des projets de fin d'année, on n'a pas été jusqu'au bout pour les fonctionnalités.

On peut ajouter la sauvegarde de partie, le chargement de partie, la génération aléatoire, et on peut mettre tout ça dans un menu graphique à côté du jeu.



\section{Portage android}
\subsection{Synthèse}
On a ajouté le dossier ``android'' qui comprend tout ce qu'il faut pour créer une application sous android de net\_sdl.
\subsection{Analyse critique}
Certains membres du groupe ont eu des difficultés avec les instructions du README qui ont fait qu'ils n'ont pas pu travailler dessus.

Pour changer ``net\_sdl.c'', on s'est à nouveau basé sur ``demo.c'', afin de changer ce qui est compilé entre gcc et ndk. On a longtemps eu des problèmes au lancement de l'apk (le jeu qui plante au démarage) sans savoir la cause, puis on a finit par comprendre que ça venait du chargement de fichier. On a alors décidé de ne pas utiliser gameio mais plutôt de recréer les pièces et directions directement dans net\_sdl, comme au début de net\_text, et le jeu n'a plus planté.

On a aussi eu des confusions au niveau du rendu, car on nous disait à la fois de ne pas inclure les bibliothèques SDL, et de pouvoir compiler directement. Il a fallu confirmation avec le professeur pour comprendre qu'il fallait suivre la première instruction.
\subsection{Bilan}
Le jeu fonctionne sous android, il reprend les mêmes fonctionnalités que net\_sdl sauf qu'on ne peut pas utiliser le solveur.
\subsection{Pistes d'amélioration}
Toutes les améliorations de net\_sdl se répercuteront sur la version android. Il faudra juste ajouter des menus sinon il n'y a pas vraiment de touches de clavier sur téléphone pour lancer des choses comme le solveur.

De plus, après avoir finit le projet, on a vu sur le forum moodle le programme qui est censé réparer le chargement de fichiers au démarrage, donc ce serait bien de rétablir default.txt.



\section{Conclusion}
\subsection{Vialar Alexandre}
Le projet était intéressant, c'était bien de nous faire travailler en groupe.

J'ai beaucoup appris sur les makefile, git, et les outils de débuggage.

Cependant ça manquait de vrais cours, ce qui fait que je passais plus de temps à réexpliquer le cours moodle à mes collègues qu'à travailler réellement.

Il serait bien aussi de mieux expliquer l'utilité de chaque cours (par exemple pour \LaTeX, personnellement je sais qu'il est utilisé par les chercheurs, par des enseignants, etc... mais c'est parce que je m'en suis déjà servi auparavant et que je m'étais renseigné dessus, tout le monde n'a pas eu cette lubie). 
J'ai d'ailleurs remarqué que d'autres étudiants ne faisaient les cours que pour la note, ne lisant pas les cours sans rendus, et ne comprenant pas l'utilité de \LaTeX.
Je pense qu'avoir ne serait-ce que 5 minutes de cours en début de TP pourrait améliorer les choses.

Le travail en autonomie est une bonne idée, mais je ne pense pas que l'apprentissage devrait se faire en autonomie, et même si un professeur passe dans les rangs, je pense que ce n'est pas suffisant.
\subsection{Walid Alouini}
Le projet est bien passé, c'était intéressant.

J'ai appris l'habitude à utiliser des codes plus volumineux que ce que j'ai fait à présent.

C'était aussi l'occasion d'apprendre d'utiliser des différents outils technologiques(git,cmake,debug...).
\subsection{Le Guen Killian}

Travailler en groupe sur ce projet à été une expérience enrichissante.

J'y ai découvert des choses intéressante tel le fonctionnement de la compilation ou le rendu graphique avec SDL2.

Ce projet en groupe aurait pu être l'occasion de présenter un logiciel de gestion de projet afin de gérer la répartion des tâches et la date de rendu.

\subsection{Gendron Mathis}

Cette UE m'a permit d'apprendre à utiliser les outils nécessaires au développement en groupe
et je suis très satisfait de l'éventuel coup d'\oe{}il que cela m'a apporté dans les pratiques
de développement des professionels.

La seule critique que j'aurais concerne les tests pour différentes parties du développement
qui comportaient des bugs ou n'étaient pas disponibles au moment du rendu desdites parties.





\section{Annexe}
\subsection{Algorithme Solveur}
\label{algosolv}
\lstinputlisting{algo_solveur.txt}
\end{document}
